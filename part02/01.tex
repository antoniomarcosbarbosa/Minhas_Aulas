%www.matematiques.com.br

\chapter{Conjuntos Numéricos}

\section{Números Naturais}
A partir da necessidade de contar as civilizações desenvolveram métodos de registros de quantidades. O pastor de ovelhas tomava uma pedrinha para cada ovelha que saia para pastar. Para cada ovelha que retornava ele retirava uma pedrinha de um local e punha em outro, caso ainda tivessem pedrinhas significa que alguma ovelha não voutou. Até hoje chamamos o processo de raciocínio numérico de calcular, o que remonta aos calculos que eram aquelas pedras utilizaram.

Com o passar do tempo métodos mais eficientes de notação foram desenvolvidos baseados em sistemas de numeração diversos. O que temos hoje é o sistema decimal e seus símbolos evoluiram para os números que vemos hoje em dia.

A sua construção teórica é basicamente o conceito de que cada número tem um sucessor e que para encontrá-lo basta somar ``um'' a ele. Chamamos o maior de sucessor e o anterior de antesessor, e a ambos chamamos de consecutivos. Além disso o único número natural que não tem um antesessor é o zero.

\begin{center}
\fbox{$\mathbb{N}=\{1,2,3, \dots\}$}
\end{center}

\section{Números Inteiros}
Podemos dizer que este conjunto é composto pelos números naturais, o conjunto dos opostos dos números naturais e o zero. Este conjunto pode ser representado por:

$$\mathbb{Z}=\{\dots, -3 -2, -1, 0, 1,2 ,3 , \dots \}$$

\subsection{Principais subconjuntos dos Inteiros}

$\mathbb{Z}^*=\{\dots, -3 -2, -1, 1,2 ,3 , \dots \}$

$\mathbb{Z}_+=\{ 0, 1,2 ,3 , \dots \}$

$\mathbb{Z}_-=\{\dots, -3 -2, -1, 0 \}$

\subsection{Explorando a subtração}




\subsection{Explorando a divisão}
O Conjunto dos Números inteiros corresponde a uma generalização dos Naturais. 

Ele é completo na soma, pórém não é completo na Divisão. Temos que deixar um resto nos inteiros. Baseado nisso temos a teori a dos restos.

Seja um número inteiro positivo N e um divisor D, teremos como resultado um quociente Q e um resto R. De tal forma que podemos representar essa divisão por:

$$N=D\times Q+R$$

Tomemos um caso particular. Costrua uma tabela em que a primeira coluna corresponde ao número N que está sendo dividido, a segunda a quantidade de parcelas (o quociente) e a terceira o resto.

\begin{table}[h]
 \centering
 \begin{tabular}{|c|c|c|c|c|}
 \hline
 Valor & Parcelas & Resto & Notação\\
 \hline
 0 & 0 & 0 & $0=0\times 3 + 0$\\
 \hline
 1 & 0 & 1 & $1=0\times 3 + 1$\\
 \hline
 2 & 0 & 2 & $2=0\times 3 + 2$\\
 \hline
 3 & 1 & 0 & $3=1\times 3 + 0$ \\
 \hline
 4 & 1 & 1 & $4=1\times 3 + 1$\\
 \hline
 5 & 1 & 2 & $5=1\times 3 + 2$\\
 \hline
 6 & 2 & 0 & $6=2\times 3 + 0$\\
 \hline
 7 & 2 & 1 & $7=2\times 3 + 1$\\
 \hline
 8 & 2 & 2 & $8=2\times 3 + 2$\\
 \hline
 \end{tabular}
 \caption{Tabela dos restos na divisão por 3}
 \label{tab:my_label}
\end{table}

Peça para os alunos continuar a tabela até o número 18. Explore os padrões.
Algumas atividades interessantes para serem feitas baseadas nessa tabela:
\newpage
\begin{enumerate}

 \item Você identificou algum padrão nos restos?
 \item Quem são os números que restam 0? O que eles tem em comum com a tabuada de 3?
 \item Quem são os números que restam 1? Qual a relação que ele tem com os múltiplos de 3?
 \item Quem são os números que restam 2? Qual a sua relação com os múltiplos de 3?
\end{enumerate}

\section{Números Racionais}

\section{Números Reais}


\section{Resumo}

\subsection{O conjunto dos números naturais N}

O mais simples. Por ser um conjunto discreto, pode ter uma representação explícita:

$\mathbb{N} = \{0, 1, 2, 3, 4, 5 \dots\}$
 
\subsection{O conjunto dos números inteiros Z}

É o que resulta da expansão de $\mathbb{N}$ na integração dos números negativos. Por ser um conjunto discreto, pode ter representação explícita: $\mathbb{Z} = \{\dots,-3, -2, -1, 0, 1, 2, 3,\dots\}.$
 
\subsection{O conjunto dos números racionais Q}

É a expansão do conjunto $\mathbb{Z}$, na qual o campo numérico passa a ocupar a parte racional da continuidade.

Por não ocupá-la completamente, é considerado um conjunto denso, sem representação explícita. 
Pode existir na reta, desde que se indiquem os espaços vazios da descontinuidade, que correspondem aos números irracionais, também à esquerda de zero.

Proponha o seguinte jogo: dois alunos tentam falar cada um um número que seja maior que o outro dentro dos números naturais. Depois fazem o mesmo mas com um limite superior aberto e depois fechado. Depois fazem a mesma coisa só que nos racionais, primeiro com limite superior fechado e depois aberto. O objetivo é que vejam que é sempre possível escolher um número maior nos racionais que não ultrapasse um certo limite aberto. 
 
\subsection{O conjunto dos números reais R}

É a expansão do conjunto $\mathbb{Q}$ na qual o campo numérico passa a ocupar toda a continuidade, graças à união dos campos racional e irracional. Por se tratar de um conjunto contínuo, não tem representação explícita. É um conjunto numérico que ocupa todos os pontos da reta, também à esquerda de zero.

%%%%%%%%%%%%%%%%%%%%%%%%%%%%%%%%%%%%%%%%%%%%
