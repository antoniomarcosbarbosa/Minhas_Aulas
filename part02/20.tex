\chapter{Geometria Analítica}

\section{Distância de Ponto a Reta}

Em um plano cartesiano $XOY$ temos uma reta de equação geral 

\begin{equation}\label{1.1}
s:ax+by+c=0
\end{equation}

 e um determinado ponto $P=(x_0,y_0)$. Temos assim que a distância do ponto a reta:

\begin{equation}\label{1.2}
d(P,r)=\dfrac{|ax_0+b_0+c|}{\sqrt{a^2+b^2}}.
\end{equation}

Note que a equação da reta $s$ também pode ser representada da fórma 

\begin{equation}
s:ax+by=-c
\end{equation}

tal que 

\begin{equation}
(a,b)\cdot (x,y)=-c
\end{equation}

é o produto escalar entre o vetor diretor da reta $(a,b)$ e um ponto genérico $(x,y)$. Vendo dessa forma o numerador da equação \eqref{1.2} se torna a aplicação da equação da reta \eqref{1.1} no ponto $P$ dividido pelo módulo do vetor diretor.
\subsection{Demonstração}

Seja $r:ax+by+c=0$ uma reta e o ponto $P=(x_0,y_0)$ um ponto fora da reta. A distância entre a o ponto e a reta dada será a distância entre o ponto $P$ e algum outro ponto $P_i=(x_i,y_i)$ tal que esse ponto é o ponto da reta $r$ que está mais próximo de $P$. \textbf{A reta determinada pelos pontos $P$ e $P_i$, chamaremos de reta $s$ é perpendicular com a reta $r$}\footnote{Demonstrar isso}.

A distância entre o ponto e a reta se resume a distância entre os pontos $P$ e $P'$ que é dado por

\begin{equation}\label{1.5}
d(P_i,P)=\sqrt{(x_i-x_0)^2+(y_i-y_0)^2}
\end{equation}

Como dois vetores ortogonais tem produto escalar seja nulo teremos que o vetor diretor da reta $s$ é $(b,-a)$ (verifique).

A equação da reta $s$ será dada por $(b,-a)(x,y)+d=0$ ou ainda

\begin{equation}\label{1.6}
    s:bx-ay+d=0.
\end{equation}

Sabendo disso vamos em busca de determinar as coordenadas de $P_i$.

Como a reta $s$ passa pelo ponto $P$, logo 

\begin{equation}\label{1.7}
\begin{split}
bx_0-ay_0+d&=0\\
    d&=ay_0-bx_0
\end{split}
\end{equation}

Guarde bem esse resultado. Utilizaremos $d$ durente os cálculo e quando for necessário iremos substituir pelo seu valor acima.

Como $r$ e $s$ são perpendiculares e se interseccionam-se no ponto $P_i$, temos

\begin{equation}\label{1.8}
\begin{cases}
ax+by+c&=0\\
bx-ay+d&=0.
\end{cases}    
\end{equation}

Isolando o valor de $y$ da segunda equação e substituindo na primeira, temos

\begin{equation}\label{1.9}
    ax+b\dfrac{bx+d}{a}+c=0.
\end{equation}

Multiplicando tudo por $a$ e isolando $x$ do lado esquerdo da igualdade teremos

\begin{equation}\label{1.10}
    x=-\dfrac{ac+bd}{a^2+b^2}.
\end{equation}

Substituindo o valor de $x$ encontrado na primeira equação do sistema \eqref{1.8} e isolando o valor de $y$, temos

\begin{equation}\label{1.11}
    y=\dfrac{ad-bc}{a^2+b^2}.
\end{equation}

Assim o ponto de intersecção entre as retas é

\begin{equation}\label{1.12}
P_i=\left(-\dfrac{ac+bd}{a^2+b^2},\dfrac{ad-bc}{a^2+b^2} \right).
\end{equation}

Substituindo as coordenadas de $P=(x_0,y_0)$ e de $P_i$ em na equação \eqref{1.5} teremos

\begin{equation}
d(P,P_i)=\sqrt{\left(x_0+\dfrac{ac+bd}{a^2+b^2} \right)^2+\left(y_0-\dfrac{ad-bc}{a^2+b^2} \right)^2}.
\end{equation}

Substituindo agora o valor de $d$ da equação \eqref{1.7} teremos

\begin{equation}
d(P,P_i)&=\sqrt{\left(x_0+\dfrac{ac+b(ay_0-bx_0)}{a^2+b^2} \right)^2+\left(y_0-\dfrac{a(ay_0-bx_0)-bc}{a^2+b^2} \right)^2}.
\end{equation}

Efetuando a distribuição dentro de cada parentesis e deixando tudo dentro de cada grande parêntesis em uma única fração poderão ser anulados alguns termos. Após isso note que o primeiro grande parêntesis permite colocar $a^2$ em evidência e no segundo $b^2$ permitindo ainda um agrupamento posterior $a^2+b^2$.

Com duas ou três manipulações algébricas pode-se converter o resultado na equação \eqref{1.2}. Será deixado como atividade para o leitor.





\section{Posições Relativas de uma Reta e uma Circunferência}

Imagine um circulo que tem o centro em $P$ e um certo raio $r$. Para todos os pontos que estão dentro do círculo sabemos que sua distância em relação a $P$ é menor que o raio. Para os que estão fora sabemos que sua distância em relação a $P$ é maior. Para os que estão exatamente sobre o círculo sua distância é exatamente igual ao raio.

Imagine agora uma série de círculos concentricos em $P$ com diversos raios. Agora imagine que uma reta $s$ está posicionado no mesmo plano que $P$ e dos círculos concêntricos. Note que alguns círculos não tocam a reta, isso significa que todos os pontos da reta estão mais distântes de $P$ que o seu raio. Note também que para alguns círculos maiores que intersectam a reta que os pontos da reta fora do círculo tem distância maior em relação a $P$ que o raio e que os que estão interiores ao círculo tem distância menor que o raio e que os dois pontos de intersecção entre o círculo e o raio tem a mesma distância que o raio.

Vamos chamar de $r_1$ o raio do círculo que não toca da reta e $r_2$ o raio do círculo que intersecta a reta em dois pontos. O raio do círculo que toca a reta em um único ponto (raio do circulo tangente a reta) é menor que $r_1$, está entre $r_1$ e $r_2$ ou é maior que $r_2$?

Ora, se for menor que $r_1$ o círculo não tocará na reta, se for maior que $r_2$ ele sempre continuará tocando em dois pontos. Sobra-nos apenas uma opção, a de que o raio do círculo em questão é maior que o do que não toca e menor do que toca em dois pontos.

Seja $r$ a reta que contém o raio do círculo tangente a reta $s$ ele irá conter o ponto $P$. Essa reta será tangente a 

\section{Ângulo entre vetores}

O ângulo entre os vetores $\vec{u}$ e $\vec{v}$ é o mesmo que o ângulo entre seus versores, já que seus versores tem a mesma dimensão e sentido. Logo

\begin{equation}\label{1.13}
\cos \theta =\dfrac{\vec{u}\cdot \vec{v}}{||\vec{u}||\cdot ||\vec{v}||}.
\end{equation}

Como temos que o cosseno se limita ao intervalo $[-1,1]$, ou seja $|\cos \theta |\leq 1$, temos que

\begin{equation}\label{1.14}
||\vec{u}||\cdot ||\vec{v}||\leq \vec{u}\cdot \vec{v}
\end{equation}

que é a desigualdade de Cauchy-Shwarzs. Nós a utilizamos para demonstrar a desigualdade triangular.

Olhando novamente para a equação \eqref{1.13} temos o produto escalar de dois versores. Ou seja

\begin{equation}\label{1.15}
\begin{split}
\cos \theta&=\dfrac{\vec{u}}{||\vec{u}||}\cdot \dfrac{\vec{v}}{||\vec{v}||}\\
&=\hat{u}\cdot \hat{v}
\end{split}
\end{equation}

\section{Alinhamento de três pontos}