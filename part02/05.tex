%%%%%%%%%%%%%%%%%%%%%%%%%%%%%%%%%%%%%%%%%%%%%%%%%%%%%%%
\chapter{Função Afim}
%http://www.unifal-mg.edu.br/matematica/?q=pifc-funcao-progressao
Vamos relembrar a tabuada de 3.


\begin{table}[h]
\begin{center}
 \begin{tabular}{|c|c|c|c|c|c|c|c|c|c|}
 \hline
 1 & 2 & 3 & 4 & 5 & 6 & 7 & 8 & 9 & 10 \\
 \hline
 & & & & & & & & & \\
 \hline
 \end{tabular}
 \end{center}
 \caption{Múltimos de 3}
 \label{tab:my_label}
\end{table}

Coloque os números de 1 a 10 sobre um eixo horizontal das abcissas e os valores encontrados sobre o eixo das ordenadas. Trace os pontos e depois a reta. Faça outra tabela e repita o processo para os múltiplos de outros valores como 2, 4 e 5.

\begin{itemize}
 \item Relação com Progressão Aritmética
 \item Tabela Entrada e Saída
 \item Pontos no Plano Cartesiano
 \item Gráfico é uma Reta
 \item Coeficiente Ângular e Linear
 \item Domínio e Imagem
 \item Equação da Reta
 \item Zeros da Função
 \item Estudo de Sinal
\end{itemize}


