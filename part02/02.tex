\chapter{Revisão}
\section{Operações}
A adição é uma das primeiras operações que aprendemos. Ela normalmente trás a ideia de união de quantidades. Mas devemos ter cuidado que a adição também pode vir mascarada de outras fórmas como a ideia de ``falta'', normalmente associada a subtração.

A soma é uma operação que tem propriedades tais como a associação, comutatividade, neutro e simétrico.

\section{Frações}

\section{Razões, Proporções e Porcentagem}

\section{Potenciação e Radiciação}

\subsection{Potencia de expoente Natural}
\begin{center}
\fbox{$ a^n=a \cdot a \cdot \dots a$}
\end{center}

\textit{n} é o expoente e \textit{a} é a base.

Exemplos
\begin{align*}
3^3&=\\
2^4&=\\
(-3)^3&=\\
(-3)^4&=
\end{align*}

O que ocorre quando o expoente é par? E quando é impar?
\begin{center}

\fbox{\begin{cases}
a^0=1 \\ a^1=a
\end{cases}}

\end{center}

\subsubsection{Propriedades}

\begin{tabular}{lrl}

 P1 & $a^n \cdot a^m=$ &$a^{n+m}$ \\

 P2 & $\dfrac{a^m}{a^n}=$ &$a^{m-n}$\\

 P3 & $(a^m)^n=$&$a^{m\cdot n}$ \\

 P4 & $ (a \cdot b)^m =$&$a^m \cdot b^m $ \\

 P5 & $ \left( \dfrac{a}{b} \right )^m =$&$ \dfrac{a^m}{b^m}$\\

\end{tabular}
\\\\\\\\
Exemplo:
\begin{align*}
 3^3 \cdot 3^2&=\\
 \dfrac{2^4}{2^3}&=\\
 (4 \cdot 3)^3&=\\
 (4^2)^2&=\\
 \left( \dfrac{2}{3} \right)^2&=\\
 (5^4 \cdot 5^6)\div 5^{10}&=\\
 \dfrac{2^{x+5}}{2^x}-\dfrac{2^{x-2}}{2^x}&=
\end{align*}

\subsubsection{Potência de Expoente Inteiro Negativo}

OBSERVAÇÃO: Inverter a base e trocar o expoente do sinal do expoente. Ou seja:
\begin{center}
\fbox{ $a^{-n}=\left(\dfrac{1}{a} \right)^n$}
\end{center}

Exemplos:
\begin{align*}
 2^{-3}&=\\
 4^{-2}&=\\
 \left(\dfrac{1}{4} \right)^{-3}&=\\
 \sqrt{3}^{-2}&=\\
\end{align*}

\subsubsection{Potência de Expoente Racional}


\begin{center}
\fbox{$\sqrt[n]{a^m}=a^{\frac{m}{n}}$}
\end{center}

Exemplos:

\begin{align*}
 \sqrt[5]{128}&=\\
 \sqrt{1024}&=\\
\end{align*}

Propriedades:

\begin{tabular}{lrl}

 P1 & $\sqrt[n]{a} \cdot \sqrt[n]{b}=$ &$\sqrt[n]{ab}$ \\

 P2 & $(\sqrt[n]{a^m})^p=$ & $\sqrt[n]{a^{mp}}$\\

 P3 & $\sqrt[x]{\sqrt[y]{a}}=$ & $\sqrt[xy]{a}$ \\

\end{tabular}

\subsubsection{Equações Exponenciais}

Exemplo
\begin{equation}
 5^x-5^{2-x}=24\\
\end{equation}

\begin{itemize}
 \item Gráfico
 \item Estudo de sinal
 \item Base diferente de zero, diferente de um, positiva
 \item Estudo de dois casos, base entre zero e um e base maior que um
 \item Estudo do gráfico de $f(x)=2^{ax+b}+c$ quando se varia a, b e c um por vez.
\end{itemize}

