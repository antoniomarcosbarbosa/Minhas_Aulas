%%%%%%%%%%%%%%%%%%%%%%%%%%%%%%%%%%%%%%%%%%%%%%%%%
\chapter{Logarítmos}

%http://www.ime.unicamp.br/~chico/ma091/ma091_2014_34_manipulacao_de_logaritmos.pdf
Enquanto a função exponencial trabalha com o resultado das potências os logaritmos trabalha no plano dos expoentes. Nisso devemos anteriormente lembrar das propriedades de produto de potências, divisões e as outras pois terão respaldo nos logaritmos.

\section{Definição}
\section{Condições de Existência de um Logaritmo}
\section{Propriedades do logaritmo}

A função logaritma é a função inversa da função exponencial. Logo é natural deduzir as propriedades e leis de construção dos logaritmos apartir dos expoentes. 

\subsection{$\log_a (1)=0$}

Esse logaritmo é o caso geral da função $f(x)=a^0$. Temos aqui uma das propriedades de potências. Para todo valor de \textit{a} o valor de $f(0)=a^0=1$. Devemos ressaltar que o termo para $a=0$ é considerada uma indeterminação matemática.

\subsection{$\log_a (a)=1$}

Aqui temos a situação quando $x=1$, ou seja $f(1)=a^1=a$.

\subsection{$\log_a (a^x)=x$}

Da definição de logaritmo temos essa propriedade. Note que se tomarmos o caso anterior basta substituir o 1 por 2 e verifiar o resultado. Depois por três. Generalizando teremos o mesmo para um x qualquer.

\subsection{$a^{\log_a (a^x)}=x$}

Neste caso basta chamar o valor de $\log_a (x)=y$. Substituindo teremos que $a^y$. Ora, pela definição de logaritmos e pela substituição que fizemos temos que $a^y=x$. Outra maneira é fazer a substituição $x=a^y$. Utilizando a propriedade anterior poderemos confirmar a afirmação. Note que as duas substituições são equivalentes já que $\log_a (x)=y \Longleftrightarrow x=a^y$.

\section{Operações com Logaritmos}

\subsection{$\log_a (xy)=\log_a (x)+\log_a (y)$}

Essa operação conhecida como ``O produto dos logaritmandos corresponde a soma dos logaritmos'' é o motivo pelo qual os logaritmos foram criados. Essa propriedade transforma o produto de números grandes em soma de potênias. Em uma época antes da invenção de computadores com capacidade de fazerem cálculos instantaneamente como as calculadoras que cabem em nossos bolsos atualmente era o que possibilitava encontrar os resultados para problemas maritmos, astronômicos e econômicos.

Note que a propriedade do produto de potências de mesma base se equivale a essa regra. Lembre-se de observar os expoentes.

\subsection{$\log_a (\frac{x}{y})=\log_a (x)-\log_a (y)$}

\subsection{$\log_a (x^c)=c\log_a (x)$}

