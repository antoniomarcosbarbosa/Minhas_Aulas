\chapter{Grandezas Diretamente e Inversamente Proporcionais}

COMPETÊNCIA DE ÁREA 4 - CONSTRUIR NOÇÕES DE VARIAÇÃO DE GRANDEZAS PARA A COMPREENSÃO DA REALIDADE E A SOLUÇÃO DE PROBLEMAS DO COTIDIANO

\paragraph{1. “Identificar a relação de dependência entre grandezas”}
Comentário do professor: Aqui é essencial que o candidato demonstre que sabe identificar uma fórmula que relacione grandezas distintas, como a força e aceleração ou o volume e a pressão, por exemplo. Geralmente, o Enem opta por relações pouco estudadas nas escolas, justamente para que o aluno consiga chegar às fórmulas por meio da interpretação dos textos que acompanham a questão. Outra característica comum é que elas apresentem uma constante de proporcionalidade em sua composição.

\paragraph{2. “Resolver situação-problema envolvendo a variação de grandezas, direta ou inversamente proporcionais”}
Comentário do professor: Essa habilidade se baseia na aplicação de regras de três, classificadas como direta, composta ou inversa. As grandezas diretamente e inversamente proporcionais permitem resoluções pelas regras de três simples. No entanto, quando há mais de três elementos envolvidos, usa-se a composta, que se resume em algumas regras de três simples. Para desenvolver um pouco melhor o conceito, confira um exemplo de exercício no qual se usaria a regra de três composta:

“Cinco operários constroem 20 metros de um muro, trabalhando 10 horas por dia em um período de 20 dias. Quantos operários serão necessários para construir 30 metros, trabalhando 6 horas por dia durante apenas 5 dias?”

\paragraph{3. “Analisar informações envolvendo a variação de grandezas como recurso para a construção de argumentação”}
Comentário do professor: As alternativas apresentam argumentos baseados nas variações de grandeza e, a partir deles, os alunos precisam encontrar a resposta esperada por meio das próprias alternativas, ou seja, realizará cálculos matemáticos para cada uma delas, buscando a que resolveria corretamente a situação-problema.

\paragraph{4. “Avaliar propostas de intervenção na realidade envolvendo variação de grandezas”}
Comentário do professor: Mais uma vez, o Enem investe para que o candidato chegue à resolução do exercício por meio da análise de qual a proposta mais adequada para uma situação problema criada. Aqui as alternativas ganham o foco da questão.