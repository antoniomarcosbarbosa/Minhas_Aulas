%%%%%%%%%%%%%%%%%%%%%%%%%%%%%%%%%%%%%%%%%%%%%%%%%%%
\chapter{Equações do Primeiro Grau}

\section{Método da Falsa Posição}

O método da falsa posição foi utilizado pelos povos antigos para resolver uma equações. No caso de equações do primeiro grau eles executavam um processo para conseguir encontrar. Vejamos um exemplo disso.

\subsubsection{Exemplo}\textit{A uma certa quantidade foram acrescentados a sua quarta parte. Obtêve-se por fim $15$ unidades. Que quantidade é essa?}

O primeiro passo é reescrever a frase em linguajar matemático. A quantidade desconhecida será representado por $x$. A quarta parte dessa quantidade será $\dfrac{x}{4}$. 
\begin{align}
    x+\dfrac{x}{4}&=15
\end{align}

Depois que foi reescrito escolhese um valor de $x$ que chamaremos de $x_0$ que será uma resposta temporária para o nosso problema. Tomemos a precaução de escolher um valor para facilitar as contas. Digamos que $x_0=4$.

\begin{align}
x_0+\frac{x_0}{4}&=x_1\\
    4+\frac{4}{4}&=4+1\\
    x_1&=5
\end{align}

Agora basta triplicar a expressão inteira.

\begin{align}
x+\frac{x}{4}&=15\\
3x_0 +\frac{3\times x_0}{4}&=3\times 5
\end{align}

Assim verificamos que o valor procurado é $x=4\times 3 = 12$.

Varifique para $x_0=8$ e depois $x_0=12$.

\section{Método da Tentativa e Erro}

Aqui iremos chutar valores para o número misterioso e registrar o resultado. Apartir do comportamento do resultado iremos rastrear a resposta.

Para exemplificar verifiquemos o seguinte problema envolvendo equação do primeiro grau.

\begin{exe}
Ao pagar três cafezinhos e um sorvete com uma nota de R\$10,00, João recebeu R\$1,20 de troco. Se o sorvete custa R\$1,60 a mais que cada cafezinho, qual é, em reais, o preço de um cafezinho?
\end{exe}

Apesar de podermos calcular o valor dessa questão atribuindo uma variável ao valor do café e equacionando os dados para obter o seu valor, iremos chutar o valor do café e então encontrarmos a resposta a partir dele.

\begin{table}[h]
\centering
\begin{tabular}{c|c |c}
   Preço do café  & Preço do sorvete & Valor da conta \\
   \hline
  $1,00$   & $2,60$  & $5,60$\\
  $1,10$   & $2,70$  & $6,00$\\
  $1,20$   & $2,80$  & $6,40$\\
  \hline
\end{tabular}
\end{table}

Note que almentando em 0,10 o valor do café teremos um aumento de 0,40 no valor da conta. Podemos então aumentar progressivamente até obtermos o valor da conta.

Nesse tipo de situação não é tão complicado equacionar a solução. Porém em provas que o objetivo não mede apenas a capacidade de resolução de problemas mas também a eficácia em um tempo curto, pensar em um método de resolver o problema de maneira mais rápida é essencial.

Recomendo que se siga as quatro etapas para a resolução de problemas conforme Polya:
\begin{enumerate}
\item Compreensão do problema: Para compreender um problema é necessário estimular o aluno a fazer perguntas:
\begin{itemize}
\item O que é
solicitado? 
\item Quais são os dados? Quais são as condições? 
\item É possível satisfazer as condições?
\item 
Elas são suficientes ou não para determinar a solução? Faltam dados? 
\item Que relações posso
estabelecer para encontrar os dados omitidos? 
\item Que fórmulas e/ou algoritmos posso utilizar?
\end{itemize}
Neste processo de compreensão do problema, muitas vezes torna-se necessário construir
figuras para esquematizar a situação proposta, destacando valores, correspondências e uso da
notação matemática.

\item Construção de uma estratégia de resolução: É importante estimular o aluno a buscar conexões entre os dados e o que é solicitado, estimulando, também, que pensem em situações similares, a fim de que possam estabelecer um plano de resolução, definindo prioridades e, se necessário, investigações complementares
para resolver o problema.

\item Execução de uma estratégia escolhida: Esta etapa é o momento de “colocar as mãos na massa”, de executar o plano idealizado. Se as etapas anteriores foram bem desenvolvidas, esta será, provavelmente a etapa mais fácil do processo de resolução de um problema. Para que o aluno obtenha sucesso, deve ser estimulado a realizar cada procedimento com muita atenção, estando atento a cada ação desenvolvida, verificando cada passo. O aluno também deve ser estimulado a mostrar que cada procedimento realizado está correto, possibilitando a afirmação de seu aprendizado e a comunicação de sua produção.

\item Revisão da solução: A revisão é um momento muito importante, pois propicia uma depuração e uma abstração da solução do problema. A depuração tem por objetivo verificar os procedimentos utilizados, procurando simplificá-los ou, buscar outras maneiras de resolver o problema de forma mais simples. A abstração tem por finalidade refletir sobre o processo realizado procurando descobrir a essência do problema e do método empregado para resolvê-lo, de modo a favorecer uma transposição do aprendizado adquirido neste trabalho para a resolução de
outras situações-problema.
\end{enumerate}





