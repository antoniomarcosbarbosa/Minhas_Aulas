%%%%%%%%%%%%%%%%%%%%%%%%%%%%%%%%%%%%%%%%%
\chapter{Física 1}
%http://sites.ifi.unicamp.br/f128/aulas/
%http://sites.ifi.unicamp.br/f128/ementa/
\section{Fórmulas Matemáticas}

\subsection{Movimento Retilíneo}

\begin{equation}\label{14.1}
    s=s_0+vt
\end{equation}

\subsection{Vetores}

O que é uma grandeza escalar?

Uma grandeza escalar é somente magniture; ela é apenas o número, positivo ou negativo.

O que é uma grandeza vetorial?

Uma grandeza vetorial é a junção de magniture com direção. Por exemplo, um cor se move para a direção sul a 40 km/h tem uma velocidade vetorial de 40km/h apontando para o sul.

SOMA VETORIAL

DESCRIÇÃO DE FUNÇÕES TRIGONOMÉTRICAS NO TRIÂGULO RETÂNGULO

COMPONENTES RETANGULARES DE UM VETOR.

ADIÇÃO DE VETORES POR SUAS COMPONENTES RETANGULARES.

PRODUTO DE VETOR POR ESCALAR

VETORES EM TRÊS DIMENSÕES: PRODUTO ESCALAR E VETORIAL

\subsection{Equilíbio de Forças Concorrentes}

CORDAS, NÓS E POLIAS SEM FRICÇÃO

FRICÇÃO E PLANOS INCLINADOS

\subsection{Cinemática Unidimencinal}

PROBLEMAS DE ACELERAÇÃO CONSTANTE

\begin{exe}
Um corpo com velocidae inicial de 8m/s se move em linha reta com aceleração constante e percorre 640 metros em 40 segundos. Para o intervalo de 40 segundos, encontre \textbf{a)} a velocidade média, \textbf{b)} a velocidade final, e \textbf{c)} a aceleração.
\end{exe}

A velocidade varia linearmente em função do tempo.
\begin{equation}\label{14.2}
v=v_0+at    
\end{equation}

Integrando em função do tempo, teremos a função posição para o movimento uniformemente variado é dado por

\begin{equation}\label{14.3}
x=x_0+v_0t+\frac{1}{2}at^2   
\end{equation}

\begin{exe}
Um caminhão começa a se mover com uma aceleração constante de $5m/s^2$. Encontre a velocidade dele e a distância percorrida depois de $4$ segundos.
\end{exe}

\begin{exe}
A velocidade de um automóvel cresce uniformemetne de $6.0m/s$ para $20m/s$ enquanto percorria $70$ metros. Encontre a aceleração e o tempo gasto.
\end{exe}

Isolando o tempo de \eqref{14.2} e substituindo em \eqref{14.3}, teremos

\begin{equation}\label{14.4}
    v^2=v_0^2+\frac{1}{2}ax
\end{equation}

que é a fórmula de \textit{Torriceli}. Note que ela não envolve o tempo.

LANÇAMENTO VERTICAL

\subsection{Movimento em Duas e Três Dimensões}

\subsubsection{Movimento Circular Uniforme}

Uma partícula está em \textbf{movimento circular uniforme} se descreve uma circunferência ou um arco de circunferência com velocidade escalar constante (\textit{uniforme}). Embora a velocidade escalar não varie, o movimento é acelerado porque a velocidade muda de direção.

A função posição no círculo de raio $r$ é dado por

\begin{equation}\label{14.5}
    \vec{r}(\theta) = r(\cos \theta, \sin \theta)
\end{equation}

Derivando em relação ao tempo, teremos

\begin{equation}\label{14.6}
    \vec{v}(\theta) = r(-\sin \theta, \cos \theta) \dfrac{d\theta}{dt}
\end{equation}

Como $\dfrac{d\theta}{dt}=w$ é a velocidade angular, e $v=rw$, temos

\begin{equation}\label{14.7}
    \vec{v}(\theta) = v(-\sin \theta, \cos \theta)
\end{equation}

Derivando em relação ao tempo, novamente, temos

\begin{equation}\label{14.8}
    \vec{a}=-\dfrac{v^2}{r}(\cos \theta, \sin \theta)
\end{equation}

O que nos dá a velocidade centrípeta vetorial que sempre aponta para o sentro e com módulo $a=\frac{v^2}{r}$.

\subsection{Força e Movimento - I}

\subsubsection{A Segunda Lei de Newton}

A força resultante que age sobre um corpo é igual ao produto da massa do corpo pela sua aceleração.

\begin{exe}
Como pode haver força de colição entre um carro de massa $m$ e velocidade constante $v$ se o carro não está acelerado?
\end{exe}

É devido a variação do momento linear que é o produto da massa pela velocidade. Existe energia armazenada em forma de movimento (energia cinética) que é convertida em trabalho durante o impacto e liberada em forma de força. O momento linear é dado por 

\begin{equation}\label{14.9}
    \vec{P}=m\vec{v}
\end{equation}

e a segunda lei de Newton assuma a seguinte forma

\begin{equation}\label{14.10}
    \vec{F}=\dfrac{dP}{dt}
\end{equation}

\begin{exe}
Uma força atua em uma massa de $2kg$ e produz uma aceleração de $3m/s^2$. Qual aceleração é porduzida pela mesma força quando atua em corpos de massa \textbf{a)} $1kg$ \textbf{b)} $4kg$? \textbf{c)} Qual é a grandeza dessa força?
\end{exe}

\subsection{Força e Movimento - II}

\subsubsection{Atrito}

\begin{equation}\label{14.11}
    f_s=\mu_sF_N
\end{equation}

\begin{equation}\label{14.12}
    f_k=\mu_kF_N
\end{equation}

\subsubsection{Propriedades do atrito}
\subsubsection{Movimento Circular}

\begin{equation}\label{14.13}
    F=m\dfrac{v^2}{R}
\end{equation}



Movimento circular Uniforme
$a=\dfrac{v^2}{R}$

$F=\dfrac{mv^2}{R}$

\section{Energia Cinética e Trabalho}

\begin{exe}
Uma força de $3N$ atua ao longo de $12$ metros na mesma direção da força. Encontre o trabalho executado por ela.
\end{exe}

Para casos em que a força é contínua, basta multiplicar o seu valor pela distância percorrida e depois observar a direção do movimento, caso tenha sido na mesma direção da força então o trabalho foi positivo, caso contrário seria negativo.

Para casos em que a força é variável então será necessário somar suas contribuições infinitesimais através do percurso. Isso é feito pela seguinte integral

\begin{equation}\label{14.14}
    W=\int_i^f F ds
\end{equation}

Substituindo \eqref{14.10} na fórmula acima, temos

\begin{equation}\label{14.15}
    W=\int_i^f \dfrac{dP}{dt} ds
\end{equation}

Abrindo a expressão do Momento linear como produto da massa pela velocidade e combinando os $\dfrac{ds}{dt}$ para substituir por $dv$ teremos

\begin{equation}\label{14.16}
    W=\int_i^f mv dv= \dfrac{mv_f^2}{2}-\dfrac{mv_i^2}{2}
\end{equation}

que corresponde a variação da Energia Cinética. Perceba que a Energia Cinética é medida em Joules, que é unidade de Trabalho.

Mostre que a fórmula de Torriceli 

\begin{equation}\label{14.17}
    v_f^2=2as+v_i^2
\end{equation}
relaciona

%\end{exe}

