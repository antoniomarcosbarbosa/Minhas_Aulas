%%%%%%%%%%%%%%%%%%%%%%%%%%%%%%%%%%%%%%%%%%%
\chapter{Matemática Financeira}
\section{Razão e Proporção}

\subsection{Razão}

\begin{defi}
Sendo $a$ e $b$ dois números racionais, com $b\not=0$, denomina-se razão entre $a$ e $b$ ou razão de $a$ para $b$ o quociente do primeiro pelo segundo: $\frac{a}{b}$ ou   $a:b$.
\end{defi}

Lê-se:

\begin{itemize}
\item Razão de $a$ para $b$
\item $a$ está para $b$
\item $a$ para $b$
\end{itemize}

Nesta razão, chamamos $a$ de antecedente e $b$ de consequênte.

Uma razão pode ser entre grandezas:
\begin{itemize}
    \item da mesma espécie, como a escala em mapas.
    \item de espécies diferentes, como a velocidade média.
\end{itemize}

\begin{exe}
Um determinado automóvel apresenta consumo médio de $9km/h$ se abastecido com etanol, e $12km/h$ se abastecido com gasolina. Sabendo que os preços para os combustíveis são de $2,80$ e $3,5 R\$/l$, respectivamente, qual o combustível mais econômico para aquele veículo? R: Gasolina
\end{exe}

\begin{exe}
Em uma turma de 100 alunos $1\%$ são menina. Quantos alunos tem que sair da sala para que a quantidade de alunos sejam $98\%$ do total de alunos da turma? R:50 alunos
\end{exe}

\begin{exe}
Na sala do $1^o$ ano de um colégio há 20 rapazes e 25 moças. Encontre a razão entre o número de rapazes e o número de moças. R: 4/5
\end{exe}

Duas razões podem ser Inversas ou Equivalentes.

\begin{defi}
Inversas quando seu produto é igual a 1.
\end{defi}

\begin{exe}
$\frac{5}{3}$ é uma proporção inversa de $\dfrac{3}{5}$ pois seu produto é 1.
\end{exe}

\begin{defi}
Equivalentes quando se igualam.
\end{defi}
\begin{exe}
$\frac{8}{6}$ e $\frac{4}{3}$ pois simplificando uma encontra-se a outra.
\end{exe}

São das razões equivalentes que se originam as proporsões.

\begin{exe}
A proporção Aurea denotada por $\phi$ é um exemplo de proporção muito recorrente na artitetura grega.

Ela surge quando um segmento qualquer ($a+b$) está para sua maior parte $a$ assim como a sua maior parte $a$ está para a menor $b$. 

\begin{center}
\fbox{
$\dfrac{a}{b}=\dfrac{a+b}{a}$
}
\end{center}
\end{exe}

Transforme a proporção em uma equação do segundo grau e encontre a expressão numérica de $\phi$.
%\end{center}

\section{Divisões Direta e Inversamente proporcionais}

O número 5 equivale à metade de 10, 15 equivale à metade de 30 e a mesma relação existe entre 25 e 50. Dizemos que os números 5, 15 e 25 são diretamente proporcionais aos números 10, 30 e 50. Podemos afirmar isso porque guardam sempre a mesma proporção ou coeficiente de proporcionalidade (1/2).

\begin{equation}
k =\frac{a_1}{b_1}=\frac{a_2}{b_2}=\frac{a_n}{b_n}
\end{equation}

Considerando a distância entre duas cidades é de 200 Km. O primeiro carro viaja a 50 Km/h, o segundo a 100 Km/h e o terceiro a 200 Km/h. O primeiro fará o percurso em 4 horas; o segundo, em 3 horas; o terceiro, em apenas 1 hora. Podemos afirmar que velocidade e tempo são grandezas inversamente proporcionais. Aumentando a velocidade do automóvel reduz-se o tempo, na proporção inversa.

\begin{equation}
k = a_1.b_1 = a_2.b_2 = a_n.b_n
\end{equation}

\subsection{Divisão em partes diretamente proporcionais}

Decompondo o número “k” em “P” partes diretamente proporcionais a “b”, encontramos o resultado com a equação abaixo:

\begin{equation}
P_i = b_i \cdot \frac{k}{(b_1+...+b_n)}
\end{equation}


\paragraph{Obs.:} $1<i<n$, onde n é o número de valores com quem as partes ($P_i$) terão direta proporção.

\begin{exe}
Dividir o número 1.000 em quatro partes diretamente proporcionais a $2, 4, 6$ e $8$.
\end{exe}

Resolvendo a parte principal da equação
\begin{equation*}
\left[\frac{k}{(b_1+...+b_n)} \right]
\end{equation*}

	
$1.000 / (2 + 4 + 6 + 8) = 50$\\

Agora, os resultados:
\begin{align*}
	P1 &= (2 \times 50) = 100\\
	P2 &= (4 \times 50) = 200\\
	P3 &= (6 \times 50) = 300\\
	P4 &= (8 \times 50) = 400\\
	P1 + P2 + P3 + P4 &= 1.000
\end{align*}


\subsection{Divisão em partes inversamente proporcionais}

Segue o mesmo princípio da equação do item 1.2, mas agora invertemos os números da proporção.

\begin{equation}
P_i = b_i \times k/(1/b_1 +...+ 1/b_n )]
\end{equation}

\begin{exe}
dividir o número $1.000$ em quatro partes inversamente proporcionais a $2, 4, 6$ e $8$. 
\end{exe}

	Resolvendo a parte principal da equação $([k/(b_1+...+b_n)])$:
	
$1.000 / (1/2 + 1/4 + 1/6 + 1/8) = 960$\\

Agora, os resultados:\\
\begin{align*}
	P1 &= (1/2 \times 960) = 480\\
	P2 &= (1/4 \times 960) = 240\\
	P3 &= (1/6 \times 960) = 160\\
	P4 &= (1/8 \times 960) = 120\\
	P1 + P2 + P3 + P4 &= 1.000\\
\end{align*}

%%%%%%%%%%%%%%%

\section{Proporções}
\begin{defi}
Dizemos que $a$, $b$, $c$ e $d$ formam uma proporção quando $\dfrac{a}{b}=\dfrac{c}{d}$. Lê-se ``$a$ está para $b$ assim como $c$ está para $d$''.
\end{defi}

Adotando apenas valores não nulos para seus elementos, verificaremos algumas propriedades:

\begin{enumerate}
\item[P1] \textbf{(Propriedade Fundamental)} O produto dos meios é igual ao produto dos extremos.

\begin{align}
    \dfrac{a}{b}=\dfrac{c}{d}\Leftrightarrow ad=bc
\end{align}

\begin{exe}
Determine a quarta proporcional de 3, 8 e 12. Ou seja $3:8::12:x$
\end{exe}

\begin{exe}
A fração 3/4 está em proporção com 6/8, pois: $$\dfrac{3}{4}=\dfrac{6}{8}$$
\end{exe}

\begin{defi}
As proporções são contínuas quando os meios são iguais.
\end{defi}

\begin{exe}
Determine a média proporcional entre 2 e 18. Ou seja $2:x::x:12$.
\end{exe}

\begin{exe}
Cálcule a $3^a$ proporcional de 2 e 10. Ou seja $2:10::10:x$.
\end{exe}

\item[P2] Dada uma proporção, a soma (ou diferença) dos elementos da primeira razão está para o primeiro consequênte assim como a soma (ou diferênça).

\begin{align}
    \dfrac{a}{b}&=\dfrac{c}{d} \Leftrightarrow \dfrac{a\pm b}{a}=\dfrac{c\pm d}{c}
\end{align}

Para verificar isso basta somar (subtrair) uma unidade de cada membro da primeira igualdade.

\begin{exe}
Em uma casa trabalham pais e filhos. A renda mensal bruta dessa família é de $4.800,00$ reais. Se a razão entre a renda dos filhos e a dos pais é de $3:5$, com quanto contribuem os filhos todo mês?
\end{exe}

\begin{align*}
    \dfrac{x}{y}=\dfrac{3}{5} \Leftrightarrow \dfrac{x+y}{x}&=\dfrac{3+5}{3}\\
    \dfrac{4800}{x}&=\dfrac{8}{3}\\
    x&=1.200,00 \mbox{ reais.}
\end{align*}


\item[P3] \textbf{(Regra da Sociedade)} Dada uma proporção e suas razões são equivalentes á razão entre a soma (ou diferença) dos antecedentes pela soma (ou diferença dos consequentes). O termo $k$ é o fator de proporcionalidade.

\begin{align}
    k=\dfrac{a}{b}=\dfrac{c}{d} \Leftrightarrow k=\dfrac{a}{b}=\dfrac{a+c}{d+b}
\end{align}

Para verificar isso basta fazer produto dos meios pelos estremos na segunda igualdade.

\begin{exe}
Dois amigos decidiram fazer juntos uma aposta na loteria, sendo que um deles gastou $40,00$ reais e outro gastou $50,00$. Eles estavam com sorte e ganharam um prêmio líquido único de $180$ mil reais. Quanto cabe a cada um?
\end{exe}

\begin{align*}
    \dfrac{A}{40}=\dfrac{B}{50}&=\dfrac{A+B}{90}\\
    &=\dfrac{180.000}{90}=2.000\\
    A &=80.000 \mbox{ reais}\\
    B &= 100.000 \mbox{ reais}.
\end{align*}

\item[P4] Dada uma proporção, seu quadrado possui razões equivalentes à razão entre o produto dos antecedentes e o produto dos consequentes.

\begin{align}
\dfrac{a}{b}=\dfrac{c}{d}\Leftrightarrow \dfrac{a^2}{b^2}=\dfrac{c^2}{d^2}=\dfrac{ac}{bd}
\end{align}
\end{enumerate}
\section{Regra de três}
Consiste em completar uma proporção dada três de seus elementos.

\begin{exe}
Identifique o tipo de regra de três a ser empregada e resolva o seguinte problema. Em uma fábrica existem máquinas funcionando, ininterruptamente por $10 \ h$ por dia, durante $4$ dias, produzindo $240.000$ lápis. Tendo-se quebrado uma das máquinas e necessitando-se fornecer, em $6$ dias, $480.000$ lápis, quantas horas por dia deverão funcionar ininterruptamente as duas máquinas em operação?
\end{exe}

Note que, em relação a quantidade de horas de funcionamento, a quantidade de máquinas, dias e lápiz é, respectivamente, inversamente, inversamente e diretamente proporcional. Com isso podemos montar a fração abaixo:

\begin{align*}
    \dfrac{x}{10}&=\dfrac{3 \times 4 \times 400000}{2 \times 6 \times 240000}\\
    x&=20 \mbox{ horas}.
\end{align*}

\section{Regra da Sociedade}

\section{Sequências}
\subsection{Progressão Aritmética}
É uma sequência numérica na qual cada termo, esceto o primeiro, é igual a soma do antecessor com um número constante denominado razão.

\begin{exe}
Interpole seis meios termos aritméticos entre $-8$ e $13$.
\end{exe}

\begin{exe}
Calcule quantos números inteiros multiplos de $3$ existem entre $13$ e $247$.
\end{exe}

\begin{exe}
Determine o maior valor que pode ter a razão de uma P.A. que admite os números $32, 227$ e $924$ como termos da progressão.
\end{exe}

\subsection{Progressão Geométrica}
É uma sucessão de números reais obtida de forma que o quociente entre dois termos quaiquer consecutivos é constante.

\section{Juros Símples}

\section{Descontos Símples}

\section{Desconto Racional}

\section{Juros Compostos}

\section{Capitalização Mista}

\section{Descontos Compostos}

\section{Rendas}

%%%%%%%%%%%%%%%%%%%%%%%%%%%%%%%%%%%%%%%%%%%%%%%%%%%%%%


\subsection{Exercícios}
\begin{multicols}{2}
\begin{enumerate}
\item \textbf{(Fundação Carlos Chagas-MARE-97)} Uma grandeza X é diretamente proporcional à grandeza Y e inversamente proporcional à grandeza Z. Isso significa que se o valor de Y duplica e o de Z passa a ser a metade, o valor de X é multiplicado por:

\begin{enumerate}

\item 0,5
\item 1
\item 2
\item 4
\item 8

%%%%%%%%%%%%%%%%%%%

\end{enumerate}

\item A divisão do número 150 em duas partes diretamente proporcionais a 2 e 4 resulta, respectivamente,  em:
\begin{enumerate}


\item 50 e 100
\item 60 e 110
\item 80 e 70
\item 50, 50 e 50
\item 100 e 50
%%%%%%%%%%%%%%

\end{enumerate}
\item A divisão do número 150 em três partes diretamente proporcionais a 2, 4 e 6 resulta, respectivamente,  em:
\begin{enumerate}

\item 50 e 100
\item 10, 90 e 50
\item 25, 50 e 75
\item 100, 25 e 25
\item 50, 50 e 50


%%%%%%%%%%%%%%
\end{enumerate}
\item A divisão do número 300 em três partes diretamente proporcionais a 3, 4 e 5 resulta, respectivamente,  em:
\begin{enumerate}

\item 50, 70 e 100
\item 60, 80 e 50
\item 25, 50 e 75
\item 75, 100 e 125
\item 100, 100 e 100

\end{enumerate}
\item A divisão do número 360 em três partes inversamente proporcionais a 2 e 6 resulta, respectivamente,  em:
\begin{enumerate}

\item $60\mbox{ e }300$
\item $280\mbox{ e }80$
\item $300\mbox{ e }60$
\item $270\mbox{ e }90$
\item $200\mbox{ e }160$

\end{enumerate}
\item A divisão do número 330 em três partes inversamente proporcionais a 2, 4 e 6 resulta, respectivamente,  em:
\begin{enumerate}

\item 180, 90 e 60
\item 60, 100, 180
\item 170, 100 e 60
\item 190, 80 e 50
\item 200, 100 e 50

\end{enumerate}
\item A divisão do número 1.100 em três partes inversamente proporcionais a 2, 4 e 6 resulta, respectivamente,  em:
\begin{enumerate}

\item 1000, 50 e 50
\item 900, 600 e 300
\item 900, 450 e 150
\item 600, 300 e 200
\item 900, 300 e 100

\end{enumerate}
\end{enumerate}
\end{multicols}
\subsection{Resoluções}
\begin{enumerate}
\item Como X e Y são diretamente proporcionais, quando um aumenta, o outro aumenta na mesma proporção. Portanto, se o valor de Y duplica, X também o fará.
Resposta: letra “c”.
\item \begin{align*}
150 / (2+4) &= 25\\
P1 &= 2 \times 25 = 50\\
P2 &= 4 \times 25 = 100
\end{align*}
Resposta: letra “a”.
\item \begin{align*}
150 / (2+4+6) &= 12,5\\
P1 = 2 \times 12,5 &= 25\\
P2 = 4 \times 12,5 &= 50\\
P3 = 6 \times 12,5 &= 100\\
\end{align*}
Resposta: letra “c”.
\item \begin{align*}
300 / (3+4+5) &= 25\\
P1 = 3 \times 25 &= 75\
P2 = 4 \times 25 &= 100\\
P3 = 5 \times 25 &= 125
\end{align*}
Resposta: letra “d”.
\item \begin{align*}
\frac{360}{1/2+1/6} &= 540\\
P1 &= (1/2) \times 540 = 270\\
P2 &= (1/6) \times 540 = 90
\end{align*}
Resposta: letra “d”.
\item \begin{align*}
\frac{330}{(\frac{1}{2} +\frac{1}{4}+\frac{1}{6})} &= 360\\
P1 &= (1/2) \times 360 = 180\\
P2 &= (1/4) \times 360 = 90\\
P3 &= (1/6) \times 360 = 60
\end{align*}
Resposta: letra “a”.
\item \begin{align*}
\frac{1100}{(\frac{1}{2} +\frac{1}{4}+\frac{1}{6})} &= 1200\\
P1 &= (1/2) \times 1200 = 600\\
P2 &= (1/4) \times 1200 = 300\\
P3 &= (1/6) \times 1200 = 200
\end{align*}
Resposta: letra “d”.
\end{enumerate}

\section{Taxa máxima anual de juros e a prática da usura na legislação brasileira}

A usura é históricamente conhecida como algo impróprio nas relações econômicas. Os caldeus e romanos tinham, a seu modo, legislações que regulamentavam as relações econômicas e as subsequêntes práticas de usuta.

Com a revolução de 30 que depôs Woshington Luiz e pôs fim a república velha e da política de alternância governamental entre São Paulo e Minas Gerais, conhecida como política café com leite, Getúlio Vargasatravés do Decreto-Lei 22.626, de 7 de abril de 1933, fixou um limite para a taxa de juros.

A regra contra a usura entrou na constituição em 1934 e se tornando ilícito penal em 1938. Em 1988 a Constituição Cidadã fixa em $12\%$ a taxa em seu artigo 192 parágrafo terceiro.

Uma discussão jurídica sobre esse terceiro parágrafo se gerou. Uns apoiavam a aplicação imediata da limitação dada pelo parágrafo em questão enquanto outros afirmavam que era necessária a regulamentação do mesmo por uma lei complementar.

Porém com a lei complementar que regulou o Sistema Financeiro não refutou o decreto que instituia a proibição da usura então permanece a lei.\footnote{\href{http://www.egov.ufsc.br/portal/conteudo/o-limite-legal-\%C3\%A0-taxa-de-juros-0}{O limite legal à taxa de juros}}
\section{Simulações de empréstimos ou financiamentos envolvendo os sistemas de amortização SAC e PRICE}

\subsection{Sistema SAC}

Vamos simular o empréstimo de 10.000 u.m. por um período de 5 meses com uma taxa mensal de 10\%.

\begin{table}[h]
    \centering
    \begin{tabular}{|c|c|c|c|c|}
    \hline
       Períodos  &  Parcela & Amortização & Juros & Saldo Devedor\\
    \hline
       0  & 0 & 0 & 0 & 10.000\\
    \hline
    1     & 3.000 & 2.000 & 1.000 & 8.000\\
    \hline
    2     & 2.800 & 2.000 & 800 & 6.000\\
    \hline
    3     & 2.600 & 2.000 & 600 & 4.000\\
    \hline
    4     & 2.400 & 2.000 & 400 & 2.000\\
    \hline
    5     & 2.200 & 2.000 & 200 & 0\\
    \hline
    \end{tabular}
    \caption{10.000 u.m. por 5 meses a 10\% no Sistema SAC}
    \label{tab:my_label}
\end{table}

\subsection{Sistema PRICE}

Vamos simular o empréstimo de 10.000 u.m. por um período de 5 meses com uma taxa mensal de 10\%.

Primeiro precisamos calcular o valor da parcela que será a constante nesse sistema. Temos que:

\begin{align*}
T&=\dfrac{A_{n|i}}{a_{n|i}}\\
&=\dfrac{10.000}{\dfrac{(1+i)^n-1}{i(1+i)^n}}\\
&=10.000\times \dfrac{i(1+i)^n}{(1+i)^n-1}\\
&=10.000 \times \dfrac{0,1\cdot 1,1^5}{1,1^5-1}\\
&\cong 10.000 \times 0,263797= 2.637,97
\end{align*}

Com o valor da parcela podemos ir para a tabela.

\begin{table}[h]
    \centering
    \begin{tabular}{|c|c|c|c|c|}
    \hline
       Períodos  &  Parcela & Amortização & Juros & Saldo Devedor\\
    \hline
       0  & 0 & 0 & 0 & 10.000\\
    \hline
    1     & 2.637,97 & 1.637,97 & 1.000 & 8362,03\\
    \hline
    2     & 2.637,97 & 1.801,77 & 836,203 & 6560,263\\
    \hline
    3     & 2.637,97 & 1.981,94 & 656,0263 & 4578,3193\\
    \hline
    4     & 2.637,97 & 2.180,14 & 457,83193 & 2398,18123\\
    \hline
    5     & 2.637,97 & 2.398,15 & 239,818123 & 0,029353\\
    \hline
    \end{tabular}
    \caption{10.000 u.m. por 5 meses a 10\% no Sistema PRICE}
    \label{tab:my_label}
\end{table}
\section{Valor deparcelas em todos os sistemas de amortização estudados}

Determinar as parcelas correspondentes a situação de um empréstimo de 10.000 u.m., exigível em 3 anos, à taxa anual de $20\%$.

\begin{tabular}{|c|c|c|}
    \hline
       Sistema  &  Fórmula & Parcela\\
    \hline
       Pagamento Único (Montante)  & $T=C(1+i)^n$ & 17.280 \\
    \hline
    Americano &  $T_{ k }=\begin{cases} Ci\quad se\quad 0\le k<n \\ C(1+i)\quad se\quad k=n \end{cases}$  &  2.000 ou 12.000 \\
    \hline
    Alemão     & $T=\dfrac{Ci}{1-(1-i)^n}$ & 4.098,361 \\
    \hline
    PRICE     & $T=C \times \dfrac{i(1+i)^n}{(1+i)^n-1}$  & 4.747,253 \\
    \hline
    SAC     & $T_{k}=C\left[ \left( 1-\dfrac{k-1}{n}  \right) i+\dfrac{1}{n}  \right]$ & $\{T_n\}$ \\
    \hline
\end{tabular}

Observação: $\{T_n\}=5.333,333; 4.666,666; 4.000$
\section{Trabalho apresentado no seminário}
\subsection{Sistema do Montante}
Esse sistema de amortização é normalmente utilizado em Letras de Câmbio, Títulos Bancários, Certificados a prazo fixo com renda final.

\begin{table}[h]
    \centering
    \begin{tabular}{|c|c|c|c|c|}
    \hline
       Períodos  &  Parcela & Amortização & Juros & Saldo Devedor\\
    \hline
    0  & 0 & 0 & 0 & 30.000\\
    \hline
    1  & 0 & 0 & 3.000 & 33.000\\
    \hline
    2     & 0 & 0 & 3.300 & 36.300\\
    \hline
    3     & 39.930 & 36.300 & 3.630 & 0\\
    \hline
       \end{tabular}
    \caption{Sistema de Montante}
    \label{tab:my_label}
\end{table}

\subsection{Amortizações Definidas Variáveis}

Esse sistema de amortização equivale ao SAC (Sistema de Amortização Constante) mudando apenas os valores das amortizações. Ele é comumente utilizado para a renegociação de dívidas de cartão de crédito.

\begin{table}[h]
    \centering
    \begin{tabular}{|c|c|c|c|c|}
    \hline
       Períodos  &  Parcela & Amortização & Juros & Saldo Devedor\\
    \hline
       0  & 0 & 0 & 0 & 20.000\\
    \hline
    1     & 7.000 & 5.000 & 2.000 & 15.000\\
    \hline
    2     & 8.500 & 7.000 & 1.500 & 8.000\\
    \hline
    3     & 8.800 & 8.000 & 800 & 0\\
    \hline
       \end{tabular}
    \caption{Amortizações Variáveis}
    \label{tab:my_label}
\end{table}
\section{estimativa de rendas e despesas (provisões e depreciação)}


\begin{table}[h]
    \centering
    \begin{tabular}{c|c}
       Ano(Início)  & Aluguel Mençal \\
       \hline
       $1^o$  & 19.024,49\\
       \hline
       $2^o$ & 18.592,52\\
       \hline
       $3^o$ & 18.236,55\\
       \hline
       $4^o$ & 17.938,54\\
       \hline 
       $5^o$ & 17.690,43\\
       \hline
    \end{tabular}
    \caption{Alugueis a cada ano de contrato}
    \label{tab:my_label}
\end{table}



