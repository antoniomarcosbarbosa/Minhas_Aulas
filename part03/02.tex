%%%%%%%%%%%%%%%%%%%%%%%%%%%%%%%%%%%%%%%%%%%%%%%%%%%
\chapter{Geometria Analítica}

\begin{defi}
Círculo $C$ de centro $A\in \pi$ e raio $r>0$ é o conjunto de pontos do plano $\pi$ situados à distância $r$ do ponto $A$, ou seja $$C=\{P\in \pi ;d(P,A)=r \}$$
\end{defi}

\begin{center}
\begin{tikzpicture}[scale=1]
\coordinate [label={below right:$A$}] (A) at (2,2);
\coordinate [label={below right:$P$}] (P) at (2,3);

\draw[->] (0,-.5)--(0,4);
\draw[->] (-.5,0)--(4,0);

\draw (A) circle (1cm);
\draw[fill=black] (A) circle (0.4mm);
\draw[fill=black] (P) circle (0.4mm);
\end{tikzpicture}
\end{center}

MEDIATRIS DO SEGMENTO AB