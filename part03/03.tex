%%%%%%%%%%%%%%%%%%%%%%%%%%%%%%%%%%%%%%%%%%%%%%%%%%
\chapter{Introdução a Lógica}



Algumas recomendações:
\begin{itemize}
    \item Acesse o portal \textit{Racha Cuca}\footnote{https://goo.gl/U9V8jH} e joge diversos jogos, especialmente:
    \begin{multicols}{2}
        \begin{itemize}
        \item Quase Nada;
        \item Cubo vermelho;
        \item Pinguins Numa Fria;
        \item Missionários e Canibais;
        \item Problemas de Lógica;
        \item Jógos de Lógica;
        \item Sudoku;
    \end{itemize}
    \end{multicols}
    \item Acesse o portal \textit{Geniol}\footnote{https://www.geniol.com.br/logica/} e jogue:
    \begin{multicols}{2}
    \begin{itemize}
        \item Sudoku;
        \item Desafios de Lógica;
        \item Tangran;
        \item Encaixe Perfeito;
    \end{itemize}
    \end{multicols}
    \item Acesse o portal \textit{Questões de Concursos}\footnote{www.qconcursos.com}. Clique na aba \textit{Questões} e depois aplique Raciocínio Lógico na Disciplina e responda questões ou filtre por algum assunto de preferência primeiro.
    \item Acesse o portal \textit{Passei Direto}\footnote{www.passeidireto.com} e procure alguns materiais como:
    \begin{itemize}
        \item Raciocínio Lógico Simplificado de Sérgio Carvalho Vl 1 e Vl 2;
        \item Notas de Aulas de Lógica do professor Joselias;
    \end{itemize}
\end{itemize}

\section{Fundamentos de Lógica}

Segundo Irving Copi lógica é: ``uma ciência do raciocínio'' pois a sua ideia está ligada ao processo de raciocínio correto e incorreto que depende da estrutura dos argumentos envolvidos nele. Assim concluimos que a lógica estuda as formas ou estruturas do pensamento, isto é, seu propósito é estudar e estabelecer propriedades das relações formais entre as proposições.

\subsection{Proposições Simples e Compostas e Operadores Lógicos}

Chama-se frase a todo conjunto de palavras ou símbolos que expressam um pensamento ou uma ideia de sentido completo. 

Para as frases que transmitem afirmação de fatos ou exprimem juízos que formamos a respeito de determinados conceitos ou entes denominamos \textbf{proposição}. Esses fatos ou juízos afirmados pela proposição em questão deverão sempre ter um valor verdadeiro (V) ou um valor falto (F), senão  a frase em si não contituirá uma proposição lógica, e sim apenas uma frase.

\begin{exe}
O Profssor Antônio é bonito.
\end{exe}




\subsection{Tabelas verdade, tautologia, contadição e Contingência}

\section{Equivalência lógica e Negação de Proposições}

\subsection{Negação-Leis de Morgan (Negativa de uma proposição Composta)}
\subsection{Equivalência - Proposições Logicamente Equivalentes}

\section{Lógica de Argumentação - Diagramas e Operadores Lógicos}

\section{Implicação Lógica}

\section{Verdades e Mentiras}

\section{Diagramas de Venn (Conjuntos)}

\section{Quantificadores}

\section{Análise Combinatória}

\section{Problemas Lógicos com Dados, Figuras e Palitos}

\section{Sequências Lógicas de Números, Letras, Palávras e Figuras}

\section{Problemas Lógicos}

\section{Raciocínio Matemático}