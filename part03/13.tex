
%%%%%%%%%%%%%%%%%%%%%%%%%%%%%%%%%%%%%%%%%
\chapter{Álgebra Abstrata}

\section{Relações Aplicações e Operações}

\subsection{Relações Binárias}

\begin{defi}
Dados dois conjuntos $E$ e $F$, não vazios, chama-se produto cartesiano de $E$ por $F$ o conjunto pormado por todos os ``pares ordenados'' $(x,y)$ com $x$ em $E$ e $y$ em $F$, Costuma-se indicar o produto cartesiano de $E$ por $F$ com a notação $E\times F$ (lê-se:``$E$ cartesiano $F$''). Assim, temos:

\begin{equation*}
    E\times F=\{(x,y); x\in E \mbox{ e }y \in F\}
\end{equation*}
\end{defi}

\begin{defi}
Chama-se relação binária de $E$ em $F$ todo subconjunto $R$ de $E \times F$.
\end{defi}

\subsection{Relações de Equivalência}

\subsection{Relaçoes de Ordem}

\section{Grupos}
\paragraph{Teorema de Noethe} Para cada simetria da natureza corresponde uma lei de conservação (e vice-versa).

Transformações simétricas em relação a translação espacial implica na conservação do momentum. As do tempo na conservação de Energia. As de rotações nas de momento angular.

Outro tipo são as transformações de Gauge no Eletromagnetismo que implicam na conservação de carga.

O conjunto de operações de simetria formam um grupo e tem as seguinte propriedades:


\section{Anéis e Corpos}

\section{Anéis de Polinômios}

\subsection{Sequências}
\begin{defi}
Sequência é toda função definida no conjunto $\mathbb{N}^*$
\end{defi}

\paragraph{Igualdade:}

\paragraph{Adição:}

\paragraph{Multiplicação:}

\subsection{Sequências quase-nulas ou polinômios}

\begin{defi}
Dado um anel $A$, uma sequência $()$ sobre $A$ recebe o nome de \textit{polinômio sobre $A$} se existe um índice $r\in \mathbb{N}$ tal que $a_m=0$ para todo $m> r$.
\end{defi}

\paragraph{Proposição 1:}

\paragraph{Proposição 2:}

\subsection{Grau de um polinômio}

\begin{defi}
Seja $f=(a_i)$ um polinômio ...
\end{defi}



\subsection{Imersão de A em A[X]}



\subsection{Divisão em A[X]}



\subsection{Raízes de Polinômios}



\subsection{Polinômios sobre Corpos}

POLINÔMIOS IRREDUTÍVEIS

\begin{defi}
Seja $K$ um corpo. Dizemos que um polinômio $p\in K[X]$ é irredutível em $K[X]$ ou irredutível sobre $K$ se

\begin{enumerate}
    \item[(i)]  $p\in K$ (ou seja, $p$ não é um polinômio constante);
    \item[(ii)] Dado $f \in K[X]$, se $f|p$, então ou $f \in K^*$ ou existe $c \in K^*$ tal que $f=cp$.
\end{enumerate}
\end{defi}

Um polinômio $g \in K[K]$, não constante e não irredutível, chamase \textit{redutível} ou \textit{composto}.