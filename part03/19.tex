%%%%%%%%%%%%%%%%%%%%%%%%%%%%%%%%%%%%%%%%%%
\chapter{EDO}

\hspace{4cm}
\begin{minipage}{15cm}
Muitos dos princípios, ou leis, que regem o comportamento do mundo físico são proposições, ou relações, envolvendo a taxa segundo a qual as coisas acontecem. Expressas em linguagem matemática, as relações são equações e as taxas são derivadas. Equações contenso derivadas são \textbf{equações diferenciais}. Pot tanto, para compreender e investigar problemas envolvendo o movimento de fluidos, o fluxo de corrente elétrica em circuitos, a dissipação de calor em objetos sólidos, a propagação e a detecção de ondas sísmicas ou o aumento ou a diminuição de populações, entre muitos outros, é necessário saber alguma coisa sobre equações diferenciais.

Uma equação diferencial que descreve algum processo físico é chamada, muitas vezes, de \textbf{modelo matemático} do processo.\nocite{boyce2010equaccoes}
\end{minipage}


%Sites para plotagem de campos de direções
%http://slopefield.nathangrigg.net/
%https://www.geogebra.org/student/mW7dAdgqc




%Stewart, James. Cálculo. Vl 2.
%Guidorizi, H.L. Um curso de Cálculo. (Vl 4) 5.ed. LTC

%Dissertação de mestrado: ALGUMAS APLICAÇÕES DAS EQUAÇÕES DIFERENCIAIS
%http://www.dmejp.unir.br/menus_arquivos/1787_2011_sergio_alitollef.pdf

%Khan Academy sobre Equações Diferenciais
%https://pt.khanacademy.org/math/differential-equations

