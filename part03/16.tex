%%%%%%%%%%%%%%%%%%%%%%%%%%%%%%%%%%%%%%%%%%
\chapter{Cálculo 4}
%http://www.ime.unicamp.br/~valle/PastCourses/MA211_14/Aula21.pdf
\section{Cálculo Vetorial}

%http://petemb.ufsc.br/files/2015/03/Apostila-Calculo-Vetorial-PROTEGIDA.pdf

\section{Integrais de Linha}

É a integral da componente tangencial da força. 
https://youtu.be/BZItyzKGmi4 por volta dos 21 minutos.

As integrais de linha tem um papel fundamental na Física. Ela é responsável por generalizar as integrais que normalmente fazemos sobre os eixos para curvas espaciais.

Devido essa ``flexibilidade'' de se modificar o lugar onde se está integrando podemos encontrar contribuições infinitesimais de massa que uma função densidade dá a medida que percorremos um fio; podemos calcular o trabalho que uma força que varia a medida que age em um meio.



\theoremstyle{definition}
\newtheorem{defi}{Definição}

\begin{defi}%Integral de linha normal
Se $F$ é definida sobre uma curva lisa $C$ dada pelas Equações $$x=x(t), y=y(t), a\leq t\leq b,$$ então a \textbf{integral de linha de $f$ sobre $C$} e 

\begin{equation}\label{16.1}
\int_C f(x,y)ds =\int_a^b f(r(t)) |r'(t)|dt= \int_a^b f(x(t),y(t))\sqrt{\left( \dfrac{dx}{dt} \right)^2+\left( \dfrac{dy}{dt} \right)^2} dt
\end{equation}

\end{defi}




\begin{defi}%Integral de linha em campos vetoriais
Seja \textbf{F} um campo vetorial contínuo definido sobre uma curva lisa $C$ dada pela função vetorial $$r(t), a\leq t\leq b.$$ Então, a \textbf{integral de linha de F ao longo de C} é $$\int_C F \cdot dr = \int_a^b F(r(t))\cdot r'(t) dt$$
\end{defi}
\section{Teorema Fundamental das Integrais de Linha}

\section{Campos Vetoriais Conservativos}

\section{Mudanças de Variáveis em Integrais Múltiplas}

\section{Superfícies Parametrizadas}

\section{Integrais de Superfície}

\section{Teorema de Green}

\section{Rotacional e Divergente}

\subsection{Operador Nabla}

\subsection{Divergente como produto Escalar}

\subsection{Rotacional como Produto Vetorial}
%https://youtu.be/GGzFH3rYUKI

\section{Teorema de Stokes}



\section*{Estratégias e Exercícios}

Os passos para a resolução são basicamente quatro:

1- Divida a sua curva se necessário em trechos que sejam integráveis ou que a parametrização seja idêntica;

Semelhantemente quando falamos de vetores,  quando vamos de um ponto A para um ponto C diretamente equivale a ir de A para B e depois para C, uma integral de linha não se importa se você a integra na curva inteira ou apenas em uma parte por vês e depois some todas as integrais no final.

Isso é uma grande mão na roda, pois para que integramos precisamos que a curva seja lisa (não tenha descontinuidades, não tenha quinas nem cúspides). Caso a curva original tenha algum(ns) desses defeitos basta separá-la em seus trechos que não tem.

Outra coisa que pode fazer com que a divisão da curva em outras é a parametrização. Integrar sobre um arco de uma função trigonométrica é diferente que integrar sobre um segmento de reta. Logo se sua curva for composta por pedaços de gráficos recomenda-se que divida ela em trechos de maneira que cada trecho se parametrize apenas por uma qualidade de função.

2- Encontre uma parametrização

Nesse momento suponho que você já tenha sua integral separada de forma que a curva tenha apenas uma parametrização. Nesse momento devemos identificar qual o tipo de parametrização necessário naquele trecho da nossa curva.

3- Reescreva sua função (densidade/força/ sei lá o que.) em relação ao parâmetro.

4- Encontre o comprimento da curva em relação ao parâmetro.

\begin{enumerate}
\item Cálcule a integral de linha, onde $C$ é uma curva dada.
\end{enumerate}

\begin{enumerate}
\item[1.] $\int_C y^3 ds, \qquad C:x=t^3, y=t, 0\leq t \leq 2$

\begin{equation}\label{16.2}
\begin{split}
\int_C y^3 ds&=\int_0^2 t^3 \sqrt{(3t^2)^2+1} dt\\
&=\int_0^2 t^3 \sqrt{9t^4+1} dt
\end{split}
\end{equation}

Tomando $u=9t^4+1$, temos que $du=36 t^3 dt$ e que para $t=0$ temos $u=1$ e para $t=2$ temos $u=145$. Assim

\begin{equation}\label{16.3}
\begin{split}
&=\dfrac{1}{36}\int_1^{145} u^{\frac{1}{2}} du\\
&=\dfrac{1}{36} \left[ \dfrac{2}{3} u^\frac{3}{2} \right]_1^{145}
\end{split}
\end{equation}

\end{enumerate}

Basicamente o que foi feito foram seguir passo a passo as seguintes orientações abaixo.

\begin{enumerate}
\item Reescreva:
$$f(x,y)=f(x(t),y(t))$$
Basta notar como os componentes ortogonais da função foram parametrizados e reescrevê-los em função desse parâmetro. Essa parte é responsável pela "densidade" ao decorrer da linha.

\item Determine: $$ds=\sqrt{\left(\dfrac{dx}{dt} \right)^2+\left(\dfrac{dy}{dt} \right)^2} dt$$
Basicamente se calculará as derivadas parciais das coordenadas ortogonais em relação ao parâmetro \textit{t} e depois tomará a raiz quadrada da soma de seus quadrados. Isso corresponderia ao módulo da derivada do parâmetro, ou seja $|r'|$. Essa parte é responsável por corrigir o ``tamânho''.
\end{enumerate}

Algo importante de nota é a forma com que a curva $C$ é definida. Aqui nessa questão ela foi dada em forma já parametrizada. Podemos ter algumas cituações difetentes para parametrizar a curva:

\begin{itemize}
\item Segmento entre dois pontos;
\item Parte de círculos;
\item Arco de uma curva entre dois pontos determinados;
\item Composição dos itens anteriores
\end{itemize}

Eventualmente precisamos dividir a curva em partes para poder calcular sua integral. Isso implica em fazer uma integral em cada local.

\begin{enumerate}
\item[2.] $\int_C xy ds \qquad C:x=t^2, y=2t, 0 \leq t \leq 1$
\end{enumerate}

Nessa segunda questão temos uma curva ainda determinada de forma paramétrica. Seguindo os dois passos indicados acima podemos encontrar rapidamente a integral que devemos calcular.

\begin{enumerate}
\item[3.] $\int_C xy^4 ds \qquad C:\mbox{a metade direita do círculo } x^2+y^2=16.$
\end{enumerate}

Aqui já encontramos um novo desafio. A curva dada não está parametrizada, teremos que fazer isso. Como a curva é uma parte de um cúrculo poderemos utilizar a parametrização básica abaixo: $$r(t)=\begin{cases} x=cos(t)\\ y=sin(t) \end{cases}, \frac{\pi}{2}\leq t \leq -\frac{\pi}{2}$$

Aqui apenas manipulamos os valores possíveis para $t$ para que ele corresponda a parte direita do círculo.

Aqui tivemos que acrescentar uma ação antes das outras duas enunciadas nos comentários da primeira questão. Os passos agora são três.

\begin{enumerate}
\item Encontre uma parametrização para a curva $C$ de forma que ela fique bem definida (seja percorrida apenas uma vês quando $t$ varia de $a$ para $b$);
\item Reescrever a função (dencidade?) $$f(x,y)=f(x(t),y(t))$$;
\item Determine: $$ds=\sqrt{\left(\dfrac{dx}{dt} \right)^2+\left(\dfrac{dy}{dt} \right)^2} dt$$
\end{enumerate}

\begin{enumerate}
\item[4.] $\int x\sin y ds, S:\mbox{ Segmento de reta entre os pontos }(0,3), (4,6) $
\end{enumerate}

Vamos encontrar uma parametrização para a curva dada. Como ela é um segmento de reta ela pode ser parametrizada pela equação da reta. Basta limitá-la em seu domínio de forma a ir apenas entre os pontos interessantes.

$$r(t)=(0,3)+t(4,6)=(4t,3+6t); 0\leq t\leq 1$$

Com isso cumprimos nossa primeira etapa. Agora vamos reescrever a função conforme a parametrização encontrada.

$$f(x,y)=x\sin y \rightarrow f(r(t))=4t \sin (3+6t) $$

O próximo passo é determinar quem é $ds/dt$.

$$ds=\sqrt{\left( \dfrac{dx}{dt} \right)^2+\left( \dfrac{dy}{dt} \right)^2}dt=\sqrt{4^2+6^2}dt=2\sqrt{13}dt$$

