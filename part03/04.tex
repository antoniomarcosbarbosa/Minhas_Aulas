\chapter{Fundamentos da Matemática 2}

\section{Trigonometria}

\begin{equation}\label{4.1}
   f(x)=a\sin(bx) 
\end{equation}


O valor de $a$ indica a amplitude e $b$ indica a frequência. O inverso da frequência é o período.

\section{Funções Periódicas}

\begin{defi}
Uma função $f$ é dita periódica se existe um número real positivo $P$ chamado período de $f$, tal que

\begin{equation}\label{4.2}
f(x)=f(x+P)
\end{equation}
\end{defi}

INCLUIR FIGURA 1: UMA FUNÇÃO PERIÓDICA

Observações:

\begin{itemize}
    \item O período $P$ é o comprimento do intervalo em $x$ necessário para a imagem da função se repetir.
    \item Segue da \eqref{4.2} que se $f$ é periódica de período $P$, então para qualquer $n$ inteiro positivo temos
    
\begin{equation}\label{4.3}
    f(x)=f(x+nP)
\end{equation}

ou seja, qualquer múltiplo inteiro positivo $nP$ DE $P$ é um período de $f$. O menor valor de $P$ que satisfaz a equação \eqref{4.2} é chamado \textit{Período Fundamental} de $f$ e será denotado por $T$. Qualquer outro período de $f$ será um múltiplo inteiro do período fundamental.

\item A frequência $F$ de uma função periódica é definida como o inverso de seu período

CONTINUA: MATERIAL SOBRE INTRODUÇÃO AS SÉRIES DE FOURIER

\end{itemize} 

\section{Números Complexos}
O que é um número complexo? O adjetivo complexo é infeliz, herdado de épocas nas quais a abstração envolvida na compreensão desses números era considerada elevada. Atualmente sabemos que o conceito de número real exige núvel de abstração equivalente e, para exemplificar isso, começamos trabalhando a mais básica ilustração que se pode dar sobre números completos: a ssolução da equação

$$X^2+1=0 \mbox{ ou, o que dá no mesmo, } X^2=-1$$

Sabemos que sobre $\mathbb{R}$ não há solução e somos forçados a definir um ``número'' $i$, satisfazendo $i^2=-1$, que resolve a equação. Agora, ou postulamos a existência desse ``número'' ou invocamos a Álgebra Linear elementar e saimos em busca de um entre de natureza geométrica que seja a solução procurada. Se assim fizermos e olharmos para essa equação sob a forma

$$X \cdot X = -I$$

onde $X$ é uma matriz $2 \times 2$ com coeficientes reais, $I=$

CONTINUAR CAPÍTULO 1 DO LIVRO DE CÁLCULO EM UMA VARIÁVEL COMPLEXA


Você compreende a expressão $e^{i\pi}=-1$\footnote{\url{https://goo.gl/MXgoNy}}?

Na expressão em questão temos a participação de quatro elementos, a base do logaritmo natural, o valor de pi e a constante imaginátia dos números complexos.

Quando derivamos uma função exponencial $a^x$ obtemos como resultado um múltiplo dela. Logo para que a derivada de uma função exponencial seja igual a ela própria basta que o valor pelo qual multiplico a minha funçã após a derivada seja $1$. Esse valor em questão é uma expressão que envolve o limite no infinito. Deduza a expressão $e=\lim_{n \to \infty} \left( 1+\dfrac{1}{n} \right)^n$ a partir da derivada da função exponencial pela definição de derivada.

Verifique que $e^x=\lim_{n \to \infty} \left( 1+\dfrac{x}{n} \right)^n$.

Note que agora substituindo o valor de $\pi$ no lugar do $x$ encontraremos algo que se aproxima muito da expressão.

$e^{\pi}=\lim_{n \to \infty} \left( 1+\dfrac{\pi}{n} \right)^n$

Agora estamos a um passo de obter o que queremos.

$e^{i\pi}=\lim_{n \to \infty} \left( 1+i\dfrac{\pi}{n} \right)^n$

Basta verificar o valor do limite acima e obter o resultado $-1$.

