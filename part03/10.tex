%%%%%%%%%%%%%%%%%%%%%%%%%%%%%%%%%%%%%%%%%%%%%
\chapter{Introdução a Teoria dos Números}
%http://www.mat.unb.br/~maierr/tnotas.pdf
%http://wwwp.fc.unesp.br/~mauri/TN/SistNum.pdf O SISTEMA TERNÁRIO É O SISTEMA DE MAIOR CAPACIDADE
\section{Resultados Preliminares}
\subsection{O princípio da indução}
\subsection{O teorema binomial}
\subsection{Os números triangulares}
\section{Teoria de divisibilidade nos inteiros}
\section{Números primos e sua distribuição}
\section{Triplos PITAGORICOS e a conjctura de FERMAT}
\section{Números deficientes-abundantes-perfeitos e de MERSENNE}

Os números primos de Mersenne são casos particulares do mito da origem do xadrez.

%https://pt.wikipedia.org/wiki/Primo_de_Mersenne

\section{A teoria das congruências}
Escolha dois competidores. Um deles escolherá dentre duas opções -- ``1'' ou ``2''-- e acrescentará uma ou duas unidades mentalmente vocalizando apenas o resultado. Ganha o primeiro que que chegar ao número ``20''.

Existe alguma forma de vencer sempre?


\subsection{Teorema Chinês do Resto}
O Teorema Chinês do Resto é a evolução do que conhecemos como Menor múltiplo comum. Ambos se equivalem se estivermos trabalhando com restos iguais a zero no Teorema Chinês do Resto.

\section{Os Teoremas de FERMAT e WILSON}




