\chapter{Tendências da Educação Matemática}
%http://www.diaadiaeducacao.pr.gov.br/portals/pde/arquivos/1785-8.pdf
%http://www.upf.br/seer/index.php/rep/article/viewFile/3506/2291
%http://revistas.fw.uri.br/index.php/revistadech/article/viewFile/303/563
%http://mat.ufrgs.br/~vclotilde/disciplinas/pesquisa/CLAUDIA_FRANCESES.DOC.pdf

\section{Etnomatemática}
Segundo D'Ambrosio (1987): Etno (sociedade, cultura, jargão, códigos, mitos, símbolos)\linebreak + matema (explicar, conhecer) + tica (tchné, arte e técnica). Raízes sócio-culturais da arte ou técnica de explicar e conhecer. A etnomatemática prioriza a cultura local onde quer que o trabalho seja desenvolvi do valorizando sempre a matemática presente nas diferentes culturas. Tem como ponto de partida o conhecimento prévio, isto é, o conhecimento adquirido com as experiências e observações fora do âmbito escolar dos alunos. Partindo dos conceitos informais trazidos pelos alunos, a etnomatemática, contraria a concepção de que todo conhecimento matemático é adquirido na escola, pois se vale desses conceitos e de situações existentes na comunidade escolar para formalizar os conceitos. O professor precisa se inteirar dos costumes, para perceber se os conceitos que os alunos têm sobre determinados assuntos são válidos, e assim saber o que pode ser mudado ou complementado. Isso exige muita disponibilidade do professor. Os principais trabalhos nesta linha são: D'Ambrosio (1986); Carraher, Carraher \& Schlieman (1988), entre outros. 
\section{Modelagem Matemática}
\section{Mídias Tecnológicas}
\section{História da Matemática}
\section{Investigação Matemática}
\section{Resolução de Problemas}

 \nocite{polya1995arte}


\thispagestyle{empty}

%\vspace{0.8cm}
\begin{footnotesize}
\begin{center}
\begin{tabular}{|cl|} \hline
\hspace{1cm} & \\
& Polya, G. (George), 1887- \\
P841a & \hspace{0.6cm}  A arte de resolver problemas: um novo aspecto do método matemático/G. Polya; \\ &tradução e adaptação Heitor Lisboa de Araújo. --2.reimp.--Rio de Janeiro: omtercoência, 1995.\\
& 196p.\\ \hline
\end{tabular}
\end{center}
\end{footnotesize}%\begin{small}
\hspace*{-1cm}


Uma grande descoberta resolve um grande problema, mas há sempre uma pitada de descoberta na resolução de qualquer problema. O problema pode ser modesto, mas se ele desafiar a curiosidade e puser em jogo as faculdades investidas, quem o resolve por seus próprios meios, experimentará a tensão e gozará o triunfo da descoberta. Experiências tais, numa idade susceptível, poderão gerar o gosto pelo trabalho mental e deixar, por toda a vida, a sua marca na mente e no caráter.

\section{Escrita e Leitura no Ensino da Matemática}

