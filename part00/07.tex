\chapter{Matrizes Curriculares}
\section*{Introdução}


\section{$5^o$ ano}
\subsection{Espaço e Forma}
\begin{enumerate}
    \item[D1] \label{itm5:D1} Identificar a localização/movimentação de objetos em mapas, croquis e outras representações gráficas.
    \item[D2] \label{itm5:D2} Identificar propriedades comuns e diferenças entre poliedros e corpos redondos, relacionando figuras tridimencionais com suas planificações.
    \item[D3] \label{itm5:D3} Identificar propriedades comuns e diferenças entre figuras bidimencionais pelo número de lados, pelos tipos de ângulos.
    \item[D4] \label{itm5:D4} Identificar quadriláteros observando as posições relativas entre seus lados (paralelos, concorrentes, perpendiculares).
    \item[D5] \label{itm5:D5} Reconhecer a conservação ou modificação de medidas dos lados, do perímetro, da área em amplicação e/ou redução de figuras poligonais usando malhas quadriculadas.
\end{enumerate}

\subsection{Grandezas e Medidas}
\begin{enumerate}
    \item[D6] \label{itm5:D6} Estimar a medida de grandezas utilizando unidades de medida convencionais ou não.
    \item[D7] \label{itm5:D7} Resolver problemas significativos utilizando unidades de medida padronizadas como $km/m/cm/mm$, $kg/g/mg, l/ml$.
    \item[D8] \label{itm5:D8} Estabelecer relações entre unidades de medida de tempo.
    \item[D9] \label{itm5:D9} Estabelecer relações entre o horário de nício e término e(ou) o intervalo da duração de um evento ou acontecimento.
    \item[D10] \label{itm5:D10} Em um problema, estabelecer trocas entre cédulas e moedas do sistema monetário brasileiro, em função de seus valores.
    \item[D11] \label{itm5:D11} Resolver problema envolvendo o cálculo do perímetro de figuras planas, desenhadas em malhas quadriculadas.
    \item[D12] \label{itm5:D12} Resolver problema envolvendo o cálculo ou estimativa de áreas de figuras planas, desenhadas em malhas quadriculadas.
\end{enumerate}
\subsection{Números e Operações}

    \begin{enumerate}
        \item[D13] \label{itm5:D13} Reconhecer e utilizar características do sistema de numeração decimal como agrupamentos e trocas na base $10$ e princípio do valor posicional.
        \item[D14] \label{itm5:D14} Identificar a localização de números naturais na reta numérica.
        \item[D15] \label{itm5:D15} Reconhecer a decomposição de números naturais nas suas diversas ordens.
        \item[D16] \label{itm5:D16} Reconhecer a composição e a decomposição de números naturais em forma polinomial.
        \item[D17] \label{itm5:D17} Calcular o resultado de uma adição ou subtração de números naturais.
        \item[D18] \label{itm5:D18} Calcular o resultado de uma multiplicação ou divisão de números naturais.
        \item[D19] \label{itm5:D19} Resolver problema com números naturais, envolvendo diferentes significados da adição ou subtração: juntar, alteração de um estado inicial (positiva ou negativa), comparação e mais de uma transformação (positiva ou negativa).
        \item[D20] \label{itm5:D20} Resolver problemas com números naturais, envolvendo diferentes significados da multiplicação ou divisão: multiplicação comparativa, ideia de proporcionalidade, configuração retangular e combinatória.
        \item[D21] \label{itm5:D21} Identificar diferentes representações de um mesmo número racional.
        \item[D22] \label{itm5:D22} Identificar a localização de números racionais representados na forma decimal na reta numérica.
        \item[D23] \label{itm5:D23} Resolver problema utilizando a escrita decimal de células e moedas do sistema monetário brasileiro.
        \item[D24] \label{itm5:D24} Identificar fração como representação que pode estar associada a diferentes significados.
        \item[D25] \label{itm5:D25} Resolver problema com números racionais expressos na forma decimal envolvendo diferentes significados da adição ou subtração.
        \item[D26] \label{itm5:D26} Resolver problema envolvendo noções de porcentagem ($25\%, 50\%,100\%$).
     \end{enumerate}
\subsection{Tratamento da Informação}
\begin{enumerate}
    \item[D27] \label{itm5:D27} Ler informações e dados apresentados em tabelas.
    \item[D28] \label{itm5:D28} Ler informações e dados apresentados em gráficos (particularmente em gráficos de colunas).
\end{enumerate}

\section{$9^o ano$}
\subsection{Espaço e Forma}
\begin{enumerate}
    \item[D1] \label{itm9:D1} Identificar a localização/movimentação de objeto, em mapas, croquis e outras representações gráficas.
    \item[D2] \label{itm9:D2} Identificar propriedades comuns e diferenças entre figuras bidimencionais e tridimencionais, relacionando-as com suas planificações.
    \item[D3] \label{itm9:D3} Identificar propriedades de triângulos pela comparação de medidas de lados e ângulos.
    \item[D4] \label{itm9:D4} Identificar relação entre quadriláteros, por meio de suas propriedades.
    \item[D5] \label{itm9:D5} Reconhecer a conservação ou modificação de medidas dos lados, do perímetro, da área em amplicação e/ou redução de figuras poligonais usando malhas quadriculadas.
    \item[D6] \label{itm9:D6} Reconhecer ângulos como mudança de direção ou giros, identificando ângulos retos e não retos.
    \item[D7] \label{itm9:D7} Reconhecer que as imagens de uma figura construída por uma transformação homotética são semelhantes, identificando propriedades e/ou medidas que se modificam ou não se alteram.
    \item[D8] \label{itm9:D8} Resolver problema utilizando a propriedade dos polígonos (soma de seus ângulos internos, número de diagonais, cálculo da medida de cada ângulo internonos polígonos regulares).
    \item[D9] \label{itm9:D9} Interpretar informações apresentadas por meio de coordenadas cartesianas.
    \item[D10] \label{itm9:D10} Utilizar relações métricas do triângulo retângulo para resolver problemas significativos.
    \item[D11] \label{itm9:D11} Reconhecer círculo/circunferência, seus elementos e algumas de suas relações.
\end{enumerate}

\subsection{Grandezas e Medidas}
\begin{enumerate}
    \item[D12] \label{itm9:D12} Resolver problema envolvendo o cálculo de perímetro de figuras planas.
    \item[D13] \label{itm9:D13} Resolver problema envolvendo o cálculo de área de figuras planas.
    \item[D14] \label{itm9:D14} Resolver problema envolvendo volume.
    \item[D15] \label{itm9:D15} Resolver problema envolvendo relações entre diferentes unidades de medida.
\end{enumerate}

\subsection{Números e Operações}
\begin{enumerate}
    \item[D16] \label{itm9:D16} Identificar a localização de números inteiros na reta numérica.
    \item[D17] \label{itm9:D17} Identificar a localização de números racionais na reta numérica.
    \item[D18] \label{itm9:D18} Efetuar cálculos com números inteiros envolvendo as operações (adição, subtração, multiplicação, divisão e potenciação).
    \item[D19] \label{itm9:D19} Resolver problemas com números naturais envolvendo diferentes significados das operações (adição, subtração, multiplicação, divisão e potenciação).
    \item[D20] \label{itm9:D20} Resolver problema com números inteiros envolvendo as operações (adição, subtração, multiplicação, divisão e potenciação).
    \item[D21] \label{itm9:D21} Reconhecer as diferentes representações de um número racional.
    \item[D22] \label{itm9:D22} Identificar fração como representação que pode estar associada a diferentes significados.
    \item[D23] \label{itm9:D23} Identificar frações equivalentes.
    \item[D24] \label{itm9:D24} Reconhecer as representações decimais dos números racionais como uma extenção do sistema de numeração decimal identificando a existência de ``ordens'' como décimos, centésimos e milésimos.
    \item[D25] \label{itm9:D25} Efetuar cálculos que envolvam operações com números racionais (adição, subtração, multiplicação, divisão e potenciação).
    \item[D26] \label{itm9:D26} Resolver problema com números racionais que envolvam as operações (adição, subtração, multiplicação, divisão e potenciação).
    \item[D27] \label{itm9:D27} Efetuar cálculos simples com valores aproximados de radicais.
    \item[D28] \label{itm9:D28} Resolver problema que envolva porcentagem.
    \item[D29] \label{itm9:D29} Resolver problema que envolva variações proporcionais, diretas ou inversas entre grandezas.
    \item[D30] \label{itm9:D30} Calcular o valor numérico de uma expressão algébrica.
    \item[D31] \label{itm9:D31} Resolver problema que envolva equação de segundo grau.
    \item[D32] \label{itm9:D32} Identificar a expressão algébrica que expressa uma regularidade observada em sequências de números ou figuras (padrões).
    \item[D33] \label{itm9:D33} Identificar uma equação ou uma inequação de primeiro grau que expressa um problema.
    \item[D34] \label{itm9:D34} Identificar um sistema de equações do primeiro grau que expressa um problema. 
    \item[D35] \label{itm9:D35} Identificar a relação entre as representações algébricas e geométricas de um sistema de equações de primeiro grau.
\end{enumerate}

\subsection{Tratamento da Informação}
\begin{enumerate}
    \item[D36] \label{itm9:D36} Resolver problema envolvendo informações apresentadas em tabelas e/ou gráficos.
    \item[D37] \label{itm9:D37} Associar informações apresentadas em listas e/ou tabelas simples aos gráficos que as representam e vice-versa.
\end{enumerate}


\section{$3^o$ médio}

\subsection{Espaço e Forma}
\begin{enumerate}
    \item[D1] \label{itm3:D1} Identificar figuras semelhantes mediante o reconhecimento de relações de proporcionalidade.
    \item[D2] \label{itm3:D2} Reconhecer aplicações das relações métricas do triângulo retângulo em um problema que envolva figuras planas ou espaciais.
    \item[D3] \label{itm3:D3} Relacionar diferentes poliedros ou corpos redondos com suas planificações ou vistas.
    \item[D4] \label{itm3:D4} Identificar a relação entre o número de vértices, faces e/ou arestas de poliedros expressa em um problema.
    \item[D5] \label{itm3:D5} Resolver problema que envolva razões trigonométricas no triângulo retângulo (seno, cosseno, tangente).
    \item[D6] \label{itm3:D6} Identificar a localicação de pontos no plano cartesiano.
    \item[D7] \label{itm3:D7} Interpretar geometricamente os coeficientes da equação de uma reta.
    \item[D8] \label{itm3:D8} Identificar a equação de uma reta apresentada a partir de dois pontos dados ou de um ponto e sua inclinação.
    \item[D9] \label{itm3:D9} Relacionar a determinação do ponto de interseção de duas ou mais retas com a resolução de um sistema de equações com duas incógnitas.
    \item[D10] \label{itm3:D10} Reconhecer entre as equações de $2^o$ grau com duas incógnitas, as que representam circunferências.
\end{enumerate}

\subsection{Grandezas e Medidas}
\begin{enumerate}
    \item[D11] \label{itm3:D11} Resolver problema envolvendo o cálculo de perímetro de figuras planas.
    \item[D12] \label{itm3:D12} Resolver problema envolvendo o cálculo de área de figuras planas.
    \item[D13] \label{itm3:D13} Resolver problema envolvendo a área total e/ou volume de um sólido (prisma, pirâmide, cilindro, cone, esfera).
\end{enumerate}

\subsection{Números e Operações}
\begin{enumerate}
    \item[D14] \label{itm3:D14} Identificar a localização de números reais na reta numérica.
    \item[D15] \label{itm3:D15} Resolver problema que envolva variações proporcionais, diretas ou inversas entre grandezas.
    \item[D16] \label{itm3:D16} Resolver problema que envolva porcentagem.
    \item[D17] \label{itm3:D17} Resolver problema que envolva equação de segundo grau.
    \item[D18] \label{itm3:D18} Reconhecer expressão algébrica que representa uma função a partir de uma tabela.
    \item[D19] \label{itm3:D19} Resolver problema envolvendo uma função de primeiro grau.
    \item[D20] \label{itm3:D20} Analisar crescimento/decrescimento, zeros de funções reais apresentadas em gráficos.
    \item[D21] \label{itm3:D21} Identificar o gráfico que representa uma situação descrita em um texto.
    \item[D22] \label{itm3:D22} Resolver problema envolvendo PA/PG dada a fórmula do termo geral.
    \item[D23] \label{itm3:D23} Reconhecer o gráfico de uma função polinomial de primeiro grau por meio de seus coeficientes.
    \item[D24] \label{itm3:D24} Reconhecer a representação algébrica de uma função do primeiro grau, dado o seu gráfico.
    \item[D25] \label{itm3:D25} Resolver problemas que envolvam os pontos de máximo ou de mínimo no gráfico de uma função polinomial do segundo grau.
    \item[D26] \label{itm3:D26} Relacionar as raízes de um polinômio com sua decomposição em fatores do primeiro grau.
    \item[D27] \label{itm3:D27} 
    \item[D28] \label{itm3:D28}
    \item[D29] \label{itm3:D29}
    \item[D30] \label{itm3:D30}
    \item[D31] \label{itm3:D31}
    \item[D32] \label{itm3:D32}
    \item[D33] \label{itm3:D33}
\end{enumerate}

\subsection{Tratamento da Informação}
\begin{enumerate}
    \item[D34] \label{itm3:D34}
    \item[D35] \label{itm3:D35}
\end{enumerate}